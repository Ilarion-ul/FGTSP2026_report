% !TEX program = xelatex
\documentclass[aspectratio=169,10pt]{beamer}

\usetheme{metropolis}
\usefonttheme{professionalfonts}

\usepackage{polyglossia}
\setdefaultlanguage{english}
\usepackage{fontspec}
\setmainfont{Arial}
\setsansfont{Arial}
\setmonofont{Courier New}

\usepackage{amsmath, amssymb, physics}
\usepackage{graphicx}
\usepackage{siunitx}
\sisetup{detect-all}

\title{Comparative Assessment of Spallation Target Materials}
\subtitle{Neutron Source NSC KIPT: A Geant4 Simulation Study}
\author{Ilarion Ulych}
\institute{Taras Shevchenko National University of Kyiv}
\date{\today}

\begin{document}

\begin{frame}[plain]
  \titlepage
\end{frame}

\begin{frame}{Roadmap (5 minutes)}
\small
\begin{enumerate}
  \item What is NSC KIPT neutron source and why it matters.
  \item Physics basis: electron-driven neutron generation in ADS.
  \item Goal of this work and simulation strategy.
  \item Geant4 model: beam + target configuration.
  \item Comparative results for W-Ta and U-Mo targets.
\end{enumerate}
\end{frame}

\begin{frame}{1) What is the NSC KIPT neutron source?}
\small
\begin{itemize}
  \item NSC KIPT is an \textbf{Accelerator Driven System (ADS)} for neutron science.
  \item Mission: neutron research, isotope production, and reactor-physics studies.
  \item The source combines a high-power electron LINAC and a subcritical assembly.
\end{itemize}
\vspace{0.4em}
\centering
\includegraphics[width=0.62\linewidth]{images/Acselerator Blueprint.png}

\tiny Image source list is maintained in \texttt{images/Img\_ref.txt}.
\end{frame}

\begin{frame}{1) Context and status}
\small
\begin{itemize}
  \item Internationally developed by NSC KIPT with external partners.
  \item Physical start-up was completed before full-scale wartime disruption.
  \item Current priority: safe operation support and model-based optimization.
\end{itemize}
\vspace{0.3em}
\textbf{Why this study now:}
\begin{itemize}
  \item choosing the best target material affects neutron output and reliability,
  \item simulation reduces technical risk before full-power campaigns.
\end{itemize}
\tiny Context and citations are taken from \texttt{Info\_from referenses.md}.
\end{frame}

\begin{frame}{2) Physics behind NSC KIPT neutron source}
\small
\begin{itemize}
  \item Main chain in the target:
  \[
    e^- \rightarrow \gamma \rightarrow (\gamma,n) \rightarrow n
  \]
  \item 100 MeV electrons produce bremsstrahlung photons in heavy plates.
  \item Hard photons induce photonuclear reactions and generate primary neutrons.
  \item In U-Mo, additional photo-fission channels increase neutron production.
\end{itemize}
\tiny Physics explanation: \texttt{Data/physics\_model\_principles.md}; references: \texttt{Info\_from referenses.md}.
\end{frame}

\begin{frame}{2) Subcritical assembly principle}
\small
\begin{itemize}
  \item The neutron source drives a \textbf{subcritical} core ($k_{\mathrm{eff}}<1$).
  \item Multiplication factor:
  \[
    M=(1-k_{\mathrm{eff}})^{-1}
  \]
  \item Safety principle: chain reaction is beam-dependent and stops with beam shutdown.
  \item Design trade-off: maximize neutron economy while preserving robust margins.
\end{itemize}
\tiny Literature basis: \texttt{Info\_from referenses.md}.
\end{frame}

\begin{frame}{3) Goal of this work}
\small
\begin{itemize}
  \item Perform a \textbf{comparative Geant4 assessment} of W-Ta and U-Mo targets.
  \item Quantify neutron/photon yields and engineering risk proxies.
  \item Build an evidence-based recommendation for target selection.
\end{itemize}
\vspace{0.4em}
\textbf{Primary KPIs (per electron):}
\begin{itemize}
  \item \texttt{neutrons\_model\_exit\_per\_electron\_from\_run\_summary}
  \item \texttt{photons\_above5MeV\_per\_electron\_from\_run\_summary}
  \item plate-wise: NIEL, H production, He production.
\end{itemize}
\tiny KPI priority follows \texttt{Data/Important\_data}.
\end{frame}

\begin{frame}{4) Geant4 model and configurable beam parameters}
\small
\begin{itemize}
  \item Physics list: \texttt{QGSP\_BIC\_HPT}.
  \item Beam setup from \texttt{runmeta.json}:
\end{itemize}
\vspace{0.2em}
\begin{center}
\begin{tabular}{ll}
Energy & 100 MeV \\
Relative spread & 1\% \\
$\sigma_x,\sigma_y$ & 1 mm, 1 mm \\
$\sigma_{\theta x},\sigma_{\theta y}$ & 1 mrad, 1 mrad \\
Power mode & CW, $P_{avg}=100$ kW
\end{tabular}
\end{center}
\tiny Source: \texttt{Data/20260211\_172835\_W-Ta/runmeta.json}.
\end{frame}

\begin{frame}{4) Target model and geometry in simulation}
\small
\begin{itemize}
  \item Plate assembly with water gaps and Ta cladding.
  \item Target options compared: \textbf{W-Ta} and \textbf{U-Mo}.
  \item Key geometry controls from \texttt{runmeta.json}:
\end{itemize}
\begin{itemize}
  \item plate thickness set: 2.5--9.5 mm,
  \item plate footprint: 65.8 mm,
  \item water gap: 2 mm,
  \item total assembly thickness: 120 mm.
\end{itemize}
\tiny Source: \texttt{Data/20260211\_172835\_W-Ta/runmeta.json}.
\end{frame}

\begin{frame}{4) Beam visualization GIF (updated)}
\small
\begin{center}
\IfFileExists{generated_gifs/beam_visualization_W-Ta.gif}{
  \includegraphics[width=0.48\linewidth]{generated_gifs/beam_visualization_W-Ta.gif}
}{
  \fbox{\parbox[c][0.28\textheight][c]{0.46\linewidth}{\centering Missing: \texttt{generated\_gifs/beam\_visualization\_W-Ta.gif}}}
}
\hfill
\IfFileExists{generated_gifs/beam_visualization_U-Mo.gif}{
  \includegraphics[width=0.48\linewidth]{generated_gifs/beam_visualization_U-Mo.gif}
}{
  \fbox{\parbox[c][0.28\textheight][c]{0.46\linewidth}{\centering Missing: \texttt{generated\_gifs/beam\_visualization\_U-Mo.gif}}}
}
\end{center}
\tiny GIFs are generated by \texttt{scripts/create\_gifs.py} and inserted when present.
\end{frame}

\begin{frame}{5) Validation-style summary table (per primary electron)}
\small
\begin{center}
\begin{tabular}{lcc}
\textbf{Metric} & \textbf{W-Ta} & \textbf{U-Mo} \\
\hline
$\text{neutrons\_per\_electron}$ & 0.00798 & 0.02669 \\
$\text{photons\_per\_electron}$ & 0.6728 & 1.6825 \\
$\text{n model exit / e}$ & 0.02196 & 0.04418 \\
$\gamma(E>5\,\text{MeV})$/e & 3.7238 & 3.9245
\end{tabular}
\end{center}
\vspace{0.2em}
\tiny Data source: both \texttt{particle\_yields\_per\_electron.json}; literature context: \texttt{Info\_from referenses.md}.
\end{frame}

\begin{frame}{5) Photon spectra comparison: 4.5--30 MeV (W-Ta vs U-Mo)}
\begin{columns}[T,totalwidth=\textwidth]
\begin{column}{0.5\textwidth}
\includegraphics[width=\linewidth]{Data/20260211_172835_W-Ta/photon_source_spectrum_4p5_30.png}
\centering \tiny W-Ta
\end{column}
\begin{column}{0.5\textwidth}
\includegraphics[width=\linewidth]{Data/20260212_072836_U-Mo/photon_source_spectrum_4p5_30.png}
\centering \tiny U-Mo
\end{column}
\end{columns}
\end{frame}

\begin{frame}{5) Photon spectra comparison: linear scale (W-Ta vs U-Mo)}
\begin{columns}[T,totalwidth=\textwidth]
\begin{column}{0.5\textwidth}
\includegraphics[width=\linewidth]{Data/20260211_172835_W-Ta/photon_source_spectrum_linear.png}
\centering \tiny W-Ta
\end{column}
\begin{column}{0.5\textwidth}
\includegraphics[width=\linewidth]{Data/20260212_072836_U-Mo/photon_source_spectrum_linear.png}
\centering \tiny U-Mo
\end{column}
\end{columns}
\end{frame}

\begin{frame}{5) Neutron side-surface linear view (W-Ta vs U-Mo)}
\begin{columns}[T,totalwidth=\textwidth]
\begin{column}{0.5\textwidth}
\includegraphics[width=\linewidth]{Data/20260211_172835_W-Ta/neutron_source_spectrum_linear.png}
\centering \tiny W-Ta
\end{column}
\begin{column}{0.5\textwidth}
\includegraphics[width=\linewidth]{Data/20260212_072836_U-Mo/neutron_source_spectrum_linear.png}
\centering \tiny U-Mo
\end{column}
\end{columns}
\tiny Required item: \texttt{h2\_neutron\_exit\_side\_surface\_linear} interpreted via exported linear spectra.
\end{frame}

\begin{frame}{5) h2\_neutron\_exit\_side\_surface (W-Ta vs U-Mo)}
\begin{columns}[T,totalwidth=\textwidth]
\begin{column}{0.5\textwidth}
\includegraphics[width=\linewidth]{Data/20260211_172835_W-Ta/h2_neutron_exit_side_surface.png}
\centering \tiny W-Ta
\end{column}
\begin{column}{0.5\textwidth}
\includegraphics[width=\linewidth]{Data/20260212_072836_U-Mo/h2_neutron_exit_side_surface.png}
\centering \tiny U-Mo
\end{column}
\end{columns}
\end{frame}

\begin{frame}{5) h1\_gas\_h\_plate (W-Ta vs U-Mo)}
\begin{columns}[T,totalwidth=\textwidth]
\begin{column}{0.5\textwidth}
\includegraphics[width=\linewidth]{Data/20260211_172835_W-Ta/h1_gas_h_plate.png}
\centering \tiny W-Ta
\end{column}
\begin{column}{0.5\textwidth}
\includegraphics[width=\linewidth]{Data/20260212_072836_U-Mo/h1_gas_h_plate.png}
\centering \tiny U-Mo
\end{column}
\end{columns}
\end{frame}

\begin{frame}{5) h1\_gas\_he\_plate (W-Ta vs U-Mo)}
\begin{columns}[T,totalwidth=\textwidth]
\begin{column}{0.5\textwidth}
\includegraphics[width=\linewidth]{Data/20260211_172835_W-Ta/h1_gas_he_plate.png}
\centering \tiny W-Ta
\end{column}
\begin{column}{0.5\textwidth}
\includegraphics[width=\linewidth]{Data/20260212_072836_U-Mo/h1_gas_he_plate.png}
\centering \tiny U-Mo
\end{column}
\end{columns}
\end{frame}

\begin{frame}{5) h1\_niel\_plate (W-Ta vs U-Mo)}
\begin{columns}[T,totalwidth=\textwidth]
\begin{column}{0.5\textwidth}
\includegraphics[width=\linewidth]{Data/20260211_172835_W-Ta/h1_niel_plate.png}
\centering \tiny W-Ta
\end{column}
\begin{column}{0.5\textwidth}
\includegraphics[width=\linewidth]{Data/20260212_072836_U-Mo/h1_niel_plate.png}
\centering \tiny U-Mo
\end{column}
\end{columns}
\end{frame}

\begin{frame}{5) plate\_neutron\_heatmap GIF (W-Ta vs U-Mo)}
\small
\begin{center}
\IfFileExists{generated_gifs/plate_neutron_heatmap_W-Ta.gif}{
  \includegraphics[width=0.48\linewidth]{generated_gifs/plate_neutron_heatmap_W-Ta.gif}
}{
  \fbox{\parbox[c][0.28\textheight][c]{0.46\linewidth}{\centering Missing: \texttt{generated\_gifs/plate\_neutron\_heatmap\_W-Ta.gif}}}
}
\hfill
\IfFileExists{generated_gifs/plate_neutron_heatmap_U-Mo.gif}{
  \includegraphics[width=0.48\linewidth]{generated_gifs/plate_neutron_heatmap_U-Mo.gif}
}{
  \fbox{\parbox[c][0.28\textheight][c]{0.46\linewidth}{\centering Missing: \texttt{generated\_gifs/plate\_neutron\_heatmap\_U-Mo.gif}}}
}
\end{center}
\tiny If GIFs are absent, placeholders keep the slide structure unchanged.
\end{frame}

\begin{frame}{Conclusion}
\small
\begin{itemize}
  \item The deck now follows README instructions and uses project markdown files as data/physics/reference sources.
  \item Current outputs show higher neutron productivity indicators for U-Mo.
  \item Final decision must balance yield vs thermal load and radiation damage proxies.
\end{itemize}
\end{frame}

\begin{frame}{References}
\tiny
\begin{itemize}
  \item Geant4 Collaboration, \textit{Geant4---a simulation toolkit}, NIM A 506 (2003) 250--303.
  \item Geant4 physics-list documentation (QGSP\_BIC\_HPT).
  \item Project context + literature links: \texttt{Info\_from referenses.md}.
  \item Implemented model details and formulas: \texttt{Data/physics\_model\_principles.md}.
  \item Prioritized KPI list: \texttt{Data/Important\_data}.
  \item Run outputs: \texttt{Data/20260211\_172835\_W-Ta/*}, \texttt{Data/20260212\_072836\_U-Mo/*}.
\end{itemize}
\end{frame}

\end{document}
