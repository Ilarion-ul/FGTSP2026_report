% !TEX program = xelatex
\documentclass[aspectratio=169,10pt]{beamer}

\usetheme{metropolis}
\usefonttheme{professionalfonts}

\usepackage{polyglossia}
\setdefaultlanguage{english}
\usepackage{fontspec}
\setmainfont{Arial}
\setsansfont{Arial}
\setmonofont{Courier New}

\usepackage{amsmath, amssymb, physics}
\usepackage{graphicx}
\usepackage{siunitx}
\sisetup{detect-all}
\usepackage{tikz}
\usetikzlibrary{arrows.meta,positioning,calc,shapes.geometric}

\title{Comparative Assessment of Spallation Target Materials}
\subtitle{Neutron Source NSC KIPT: Context and Beam-Model Foundations}
\author{Ilarion Ulych}
\institute{Taras Shevchenko National University of Kyiv}
\date{\today}

\begin{document}

\begin{frame}[plain]
  \titlepage
\end{frame}

\begin{frame}{Roadmap (aligned with presentation plan)}
\small
\begin{enumerate}
  \item Project concept, capabilities, and purpose (1--2 slides).
  \item Context, history, and current status (1--2 slides).
  \item Physical principles of operation (1--2 slides).
  \item Geant4 implementation with physics and engineering focus.
  \item Target scheme and key problematics for material comparison.
\end{enumerate}
\end{frame}

\begin{frame}{NSC KIPT neutron source: concept and strategic purpose}
\small
\begin{columns}[T,totalwidth=\textwidth]
\begin{column}{0.57\textwidth}
\begin{itemize}
  \item \textbf{Facility type:} hybrid accelerator-driven system (ADS).
  \item \textbf{Mission:} establish a modern Ukrainian base for neutron science and nuclear medicine.
  \item \textbf{Key capabilities:}
  \begin{itemize}
    \item medical isotope production ($^{99}$Mo, $^{188}$Re, $^{103}$Pd),
    \item BNCT technology development,
    \item reactor-physics and radiation-materials studies,
    \item training platform for nuclear specialists.
  \end{itemize}
\end{itemize}
\end{column}
\begin{column}{0.43\textwidth}
\centering
\includegraphics[width=0.95\linewidth]{images/Acselerator Blueprint.png}
\vspace{0.4em}

\tiny Blueprint view of accelerator section (repository asset).
\end{column}
\end{columns}
\vspace{0.2em}
\tiny Sources: Karnaukhov et al. (2024); Vodin et al. (2013); NSC KIPT (2004); IAEA (2021).
\end{frame}

\begin{frame}{Context, history, and current status}
\small
\begin{itemize}
  \item \textbf{International collaboration:} NSC KIPT (Ukraine) + Argonne National Laboratory (USA).
  \item \textbf{Technical partner:} IHEP (China) designed and manufactured the LINAC driver.
  \item \textbf{Milestones:}
  \begin{itemize}
    \item 2010--2021: design, construction, installation,
    \item 2021: successful physical start-up.
  \end{itemize}
  \item \textbf{Post-2022 status:} commissioning and start-up data analysis continue, with focus on safety and pilot-operation readiness.
\end{itemize}
\vspace{0.2em}
\tiny Sources: Gohar et al. (2022); Zelinsky et al. (2023); Aizatskiy et al. (2013); Karnaukhov et al. (2024).
\end{frame}

\begin{frame}{Physical principles I: primary neutron production}
\small
\begin{columns}[T,totalwidth=\textwidth]
\begin{column}{0.58\textwidth}
\begin{itemize}
  \item Driver LINAC: \SI{100}{MeV} electrons, average beam power up to \SI{100}{kW}.
  \item In heavy target material, electrons generate bremsstrahlung photons.
  \item High-energy photons induce $(\gamma,n)$ and (for U-containing options) photo-fission channels.
  \item Material choice trade-off:
  \begin{itemize}
    \item W-based targets: robust thermal performance,
    \item U-Mo: potentially higher neutron yield due to extra fission contribution.
  \end{itemize}
\end{itemize}
\end{column}
\begin{column}{0.42\textwidth}
\centering
\begin{tikzpicture}[node distance=5mm,>=Stealth,scale=0.82, transform shape]
  \tikzstyle{box}=[draw,rounded corners,align=center,minimum width=2.8cm,minimum height=0.8cm]
  \node[box,fill=blue!10] (e) {$e^-$ beam\\\SI{100}{MeV}};
  \node[box,fill=orange!15,below=of e] (g) {Bremsstrahlung\\$\gamma$ field};
  \node[box,fill=green!15,below=of g] (n) {$(\gamma,n)$ / photo-fission\\primary neutrons};
  \draw[->,thick] (e)--(g);
  \draw[->,thick] (g)--(n);
\end{tikzpicture}
\end{column}
\end{columns}
\vspace{0.2em}
\tiny Sources: Bezditko et al. (2017); Pei et al. (2013); Gann et al. (2021, 2024); Karnaukhov et al. (2023).
\end{frame}

\begin{frame}{Physical principles II: subcritical multiplication and safety}
\small
\begin{columns}[T,totalwidth=\textwidth]
\begin{column}{0.62\textwidth}
\begin{itemize}
  \item Subcritical assembly uses LEU fuel ($<20\%$ enrichment).
  \item Source-neutron amplification follows:
  \[
    M \approx \frac{1}{1-k_{\mathrm{eff}}}
  \]
  \item Design constraint: \(k_{\mathrm{eff}} \le 0.98\) ensures inherent safety.
  \item Chain reaction is not self-sustaining; switching off the beam stops the process.
  \item Reflector/moderator set: Be + graphite reflectors, demineralized water moderator.
\end{itemize}
\end{column}
\begin{column}{0.38\textwidth}
\centering
\begin{tikzpicture}[>=Stealth,scale=0.86, transform shape]
  \draw[fill=gray!20,rounded corners] (0,0) rectangle (3.6,2.8);
  \draw[fill=yellow!40] (1.2,0.9) rectangle (2.4,1.9);
  \node at (1.8,1.4) {SCA};
  \draw[->,thick,blue] (-0.8,1.4)--(1.2,1.4);
  \node[blue] at (-0.7,1.8) {$n_s$};
  \draw[->,thick,red] (2.4,1.4)--(4.4,1.4);
  \node[red] at (4.2,1.8) {$M\cdot n_s$};
  \node[align=center] at (1.8,0.35) {$k_{\mathrm{eff}}<1$};
\end{tikzpicture}
\end{column}
\end{columns}
\vspace{0.2em}
\tiny Sources: NSC KIPT (2004); Vodin et al. (2013); Zelinsky et al. (2023); IAEA (2021).
\end{frame}

\begin{frame}{Simulation model: Geant4 and scoring logic}
\small
\begin{itemize}
  \item Physics list baseline: \texttt{QGSP\_BIC\_HPT}; results normalized per primary electron.
  \item Primary reaction chain represented in model:
  \[
    e^- \rightarrow \gamma \rightarrow (\gamma,n) \rightarrow n
  \]
  \item Plate-level scorers: energy deposition, neutron exits/surface maps, NIEL proxy, H/He gas proxies.
  \item Compared targets in current dataset: W-Ta and U-Mo.
\end{itemize}
\vspace{0.3em}
\tiny Internal model source: \texttt{Data/physics\_model\_principles.md}.
\end{frame}

\begin{frame}{Beam model and parameters used in Geant4}
\small
\begin{columns}[T,totalwidth=\textwidth]
\begin{column}{0.58\textwidth}
\begin{itemize}
  \item One primary electron per event; central energy \(E_0\).
  \item Energy spread options: Gaussian or Uniform around \(E_0\).
  \item Spatial profile: \(x,y\) sampled from Gaussian beam spot.
  \item Angular divergence in \(x,y\), with optional halo and tilt defects.
  \item Optional emittance-based divergence estimate.
\end{itemize}
\vspace{0.2em}
\[
E = E_0(1+\delta),\quad \delta\sim\mathcal{N}(0,\sigma_{rel})\;\text{or}\;\mathcal{U}(-\Delta,\Delta)
\]
\end{column}
\begin{column}{0.42\textwidth}
\centering
\begin{tikzpicture}[>=Stealth,scale=0.78, transform shape]
  \draw[fill=blue!6,rounded corners] (0,0) rectangle (4.2,4.2);
  \node[draw,circle,fill=blue!18,minimum size=1.4cm] (src) at (1.2,2.1) {Beam};
  \node[draw,rectangle,fill=orange!20,minimum width=1.2cm,minimum height=3.2cm] (tar) at (3.25,2.1) {Target};
  \draw[->,thick] (src) -- node[above]{\tiny $E_0,\sigma_E$} (tar);
  \draw[->,thick] (1.55,2.45) -- (2.7,3.2);
  \draw[->,thick] (1.55,1.75) -- (2.7,1.0);
  \node[align=left,anchor=west] at (0.2,3.65) {\tiny $\sigma_x,\sigma_y$};
  \node[align=left,anchor=west] at (0.2,0.55) {\tiny halo, tilt};
\end{tikzpicture}
\end{column}
\end{columns}
\vspace{0.2em}
\tiny Internal model source: \texttt{Data/physics\_model\_principles.md}.
\end{frame}

\begin{frame}{Target scheme used in simulations}
\small
\begin{columns}[T,totalwidth=\textwidth]
\begin{column}{0.50\textwidth}
\centering
\begin{tikzpicture}[>=Stealth,scale=0.82, transform shape]
  \draw[fill=gray!10] (0,0) rectangle (4.7,3.9);
  \draw[fill=blue!20] (0.6,0.6) rectangle (1.5,3.3);
  \node[rotate=90] at (1.05,1.95) {Cooling / cladding};
  \draw[fill=orange!30] (1.5,0.6) rectangle (3.6,3.3);
  \node at (2.55,1.95) {Target plate stack};
  \draw[fill=green!20] (3.6,0.6) rectangle (4.2,3.3);
  \node[rotate=90] at (3.9,1.95) {Backing};
  \draw[->,thick] (-0.8,1.95)--(0.6,1.95);
  \node at (-0.9,2.35) {\tiny e$^-$ beam};
\end{tikzpicture}
\end{column}
\begin{column}{0.50\textwidth}
\begin{itemize}
  \item Layered target approximation for W-Ta and U-Mo cases.
  \item Beam incidence, plate stack, and support regions are explicitly scored.
  \item Plate-level observables: neutron yield, energy deposition, NIEL, H/He gas proxies.
\end{itemize}
\vspace{0.3em}
\tiny Internal implementation details: \texttt{Data/physics\_model\_principles.md}.
\end{column}
\end{columns}
\end{frame}

\begin{frame}{Target problematics: what limits performance?}
\small
\begin{columns}[T,totalwidth=\textwidth]
\begin{column}{0.52\textwidth}
\begin{itemize}
  \item \textbf{Heat localization:} strong non-uniform plate loading creates thermal-stress risk.
  \item \textbf{Radiation damage:} NIEL-driven degradation accumulates by plate depth.
  \item \textbf{Gas production:} He/H generation can worsen swelling and mechanical stability.
  \item \textbf{Design trade-off:} maximize neutron output while preserving target lifetime.
\end{itemize}
\vspace{0.25em}
\tiny KPI basis from \texttt{Data/Important\_data} and run outputs.
\end{column}
\begin{column}{0.48\textwidth}
\includegraphics[width=\linewidth]{Data/20260212_072836_U-Mo/plate_neutron_heatmap_1.png}
\vspace{0.15em}

\includegraphics[width=\linewidth]{Data/20260211_172835_W-Ta/h1_niel_plate.png}
\end{column}
\end{columns}
\end{frame}

\begin{frame}{References (for Part I)}
\tiny
\begin{itemize}
  \item NSC KIPT (2004). Concept and safety framework documents.
  \item Vodin et al. (2013); Pei et al. (2013); Bezditko et al. (2017).
  \item Gohar et al. (2022); Zelinsky et al. (2023); Karnaukhov et al. (2023, 2024).
  \item IAEA (2021) materials on subcritical systems and safety.
  \item Internal model and run artifacts: \texttt{Data/physics\_model\_principles.md}, \texttt{Data/Important\_data}.
\end{itemize}
\end{frame}

\end{document}
