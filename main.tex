% !TEX program = xelatex
\documentclass[aspectratio=169,10pt]{beamer}

\usetheme{metropolis}
\usefonttheme{professionalfonts}

\usepackage{polyglossia}
\setdefaultlanguage{english}
\usepackage{fontspec}
\setmainfont{Arial}
\setsansfont{Arial}
\setmonofont{Courier New}

\usepackage{amsmath, amssymb, physics}
\usepackage{graphicx}
\usepackage{animate}      % GIF/animation workflows in scientific decks
\usepackage{multimedia}   % media embedding support in beamer
\usepackage{siunitx}
\sisetup{detect-all}

\title{Comparative Assessment of Spallation Target Materials}
\subtitle{Neutron Source NSC KIPT: A Geant4 Simulation Study}
\author{Ilarion Ulych}
\institute{Taras Shevchenko National University of Kyiv}
\date{\today}

% Prefer new U-Mo dataset when available; fallback keeps compatibility.
\newcommand{\WTaDir}{Data/20260212_140400_W-Ta}
\IfFileExists{Data/20260212_100958_U-Mo/runmeta.json}{
  \newcommand{\UMoDir}{Data/20260212_100958_U-Mo}
}{
  \newcommand{\UMoDir}{Data/20260212_072836_U-Mo}
}

\begin{document}

\begin{frame}[plain]
  \titlepage
\end{frame}

\begin{frame}{Roadmap (5 minutes)}
\small
\begin{enumerate}
  \item NSC KIPT neutron source: concept and mission.
  \item Physical basis of neutron generation.
  \item Research objective and comparison logic.
  \item Geant4 implementation: beam and target engineering.
  \item Comparative results and practical implications.
\end{enumerate}
\end{frame}

\begin{frame}{What is the NSC KIPT neutron source?}
\small
\begin{itemize}
  \item NSC KIPT is an \textbf{Accelerator-Driven System (ADS)} for neutron science.
  \item Mission: fundamental research, isotope production, and applied nuclear studies.
  \item The facility combines a high-power electron LINAC with a subcritical assembly.
\end{itemize}
\vspace{0.3em}
\begin{figure}
\centering
\IfFileExists{images/nsc_kipt_facility_model1.jpg}{
  \includegraphics[width=0.76\linewidth]{images/nsc_kipt_facility_model1.jpg}
}{
  \fbox{\parbox[c][0.30\textheight][c]{0.76\linewidth}{\centering Placeholder: facility model image}}
}
\caption{Conceptual view of the NSC KIPT neutron-source facility.}
\end{figure}
\tiny Source: project references and institutional materials.
\end{frame}

\begin{frame}{Context and status}
\small
\begin{itemize}
  \item The project was developed in international collaboration with strong scientific and engineering support.
  \item Commissioning activities established the physical start-up baseline.
  \item Current stage focuses on safe operation and data-informed optimization.
\end{itemize}
\vspace{0.2em}
\begin{figure}
\centering
\IfFileExists{images/nsc_kipt_facility_model2.jpg}{
  \includegraphics[width=0.86\linewidth]{images/nsc_kipt_facility_model2.jpg}
}{
  \fbox{\parbox[c][0.30\textheight][c]{0.86\linewidth}{\centering Placeholder: large facility photograph}}
}
\caption{NSC KIPT installation in operational context.}
\end{figure}
\tiny Source: public project communications and literature references.
\end{frame}

\begin{frame}{Physics behind NSC KIPT neutron source}
\small
\begin{itemize}
  \item Neutron production chain in the target:
  \[
    e^- \rightarrow \gamma \rightarrow (\gamma,n) \rightarrow n
  \]
  \item High-energy electrons generate bremsstrahlung photons in heavy target plates.
  \item Photonuclear interactions produce primary neutrons with broad spectral distributions.
\end{itemize}
\vspace{0.1em}
\begin{columns}[T,totalwidth=\textwidth]
\begin{column}{0.5\textwidth}
\begin{figure}
\centering
\IfFileExists{images/photonuclear_spectrum_1.png}{
  \includegraphics[width=\linewidth]{images/photonuclear_spectrum_1.png}
}{
  \fbox{\parbox[c][0.18\textheight][c]{0.94\linewidth}{\centering Placeholder: photonuclear spectrum (I)}}
}
\caption{Representative photonuclear spectrum: component I.}
\end{figure}
\end{column}
\begin{column}{0.5\textwidth}
\begin{figure}
\centering
\IfFileExists{images/photonuclear_spectrum_2.png}{
  \includegraphics[width=\linewidth]{images/photonuclear_spectrum_2.png}
}{
  \fbox{\parbox[c][0.18\textheight][c]{0.94\linewidth}{\centering Placeholder: photonuclear spectrum (II)}}
}
\caption{Representative photonuclear spectrum: component II.}
\end{figure}
\end{column}
\end{columns}
\tiny Source: established photonuclear-reaction literature.
\end{frame}

\begin{frame}{Subcritical assembly principle}
\small
\begin{itemize}
  \item The source drives a \textbf{subcritical} core ($k_{\mathrm{eff}}<1$).
  \item Neutron multiplication is described by
  \[
    M=(1-k_{\mathrm{eff}})^{-1}.
  \]
  \item Beam-off condition inherently terminates the fission chain.
\end{itemize}
\begin{figure}
\centering
\IfFileExists{images/ADScore.jpg}{
  \includegraphics[width=0.72\linewidth]{images/ADScore.jpg}
}{
  \fbox{\parbox[c][0.26\textheight][c]{0.72\linewidth}{\centering Placeholder: subcritical assembly model image}}
}
\caption{Functional model of the subcritical assembly and neutron multiplication.}
\end{figure}
\tiny Source: reactor-physics references for ADS systems.
\end{frame}

\begin{frame}{Goal of this work}
\small
\begin{itemize}
  \item Perform a \textbf{comparative Geant4 assessment} of W-Ta and U-Mo targets.
  \item Quantify neutron productivity together with engineering risk proxies.
  \item Derive an evidence-based recommendation for target selection.
\end{itemize}
\vspace{0.4em}
\textbf{Primary comparison metrics:}
\begin{itemize}
  \item neutron output at model boundaries,
  \item high-energy photon intensity,
  \item plate-wise damage and gas-production indicators.
\end{itemize}
\tiny Source: simulation campaign and referenced benchmark values.
\end{frame}

\begin{frame}{Geant4 implementation: beam and accelerator conditions}
\small
\begin{itemize}
  \item Physics list optimized for hadronic and neutron transport in this energy range.
  \item Electron beam defined with realistic energy spread, spot size, and angular divergence.
\end{itemize}
\vspace{0.2em}
\begin{center}
\begin{tabular}{ll}
Beam energy & 100 MeV \\
Relative energy spread & 1\% \\
Transverse beam size ($\sigma_x,\sigma_y$) & 1 mm, 1 mm \\
Angular divergence ($\sigma_{\theta x},\sigma_{\theta y}$) & 1 mrad, 1 mrad \\
Operation mode & Continuous-wave, 100 kW average power
\end{tabular}
\end{center}
\tiny Source: simulation input configuration.
\end{frame}

\begin{frame}{Geant4 implementation: target engineering model}
\small
\begin{itemize}
  \item Layered plate assembly with cooling gaps and cladding.
  \item Candidate materials: \textbf{W-Ta} and \textbf{U-Mo}.
  \item Geometry tuned for thermal management and neutron production stability.
\end{itemize}
\vspace{0.2em}
\begin{itemize}
  \item plate thickness range: 2.5--9.5 mm,
  \item plate footprint: 65.8 mm,
  \item cooling gap: 2 mm,
  \item total target assembly thickness: 120 mm.
\end{itemize}
\tiny Source: simulation geometry model.
\end{frame}

\begin{frame}{Beam--target interaction visualization}
\small
\begin{center}
\begin{figure}[h]
\centering
  \animategraphics[loop,autoplay,controls,width=\linewidth]{2}{frames_animate/frame_}{0001}{0024}
\end{figure}
\end{center}
\vspace{-0.2em}
\begin{center}
\footnotesize Figure: Geant4 simulation visualization.
\end{center}
\tiny Source: own Geant4 simulation results.
\end{frame}

\begin{frame}{Comparative KPI summary (per primary electron)}
\small
\begin{center}
\begin{tabular}{lcc}
\textbf{Metric} & \textbf{W-Ta} & \textbf{U-Mo} \\
\hline
Neutron model-exit output & 0.02196 & 0.04418 \\
High-energy photons ($E>5$ MeV) & 3.72382 & 3.92452
\end{tabular}
\end{center}
\tiny Source: own Geant4 simulation results.
\end{frame}

\begin{frame}{Photon spectra in the 4.5--30 MeV region}
\begin{columns}[T,totalwidth=\textwidth]
\begin{column}{0.5\textwidth}
\begin{figure}
\centering
\includegraphics[width=\linewidth]{\WTaDir/photon_source_spectrum_4p5_30.png}
\caption{Photon spectrum for W-Ta target.}
\end{figure}
\end{column}
\begin{column}{0.5\textwidth}
\begin{figure}
\centering
\includegraphics[width=\linewidth]{\UMoDir/photon_source_spectrum_4p5_30.png}
\caption{Photon spectrum for U-Mo target.}
\end{figure}
\end{column}
\end{columns}
\tiny Source: own Geant4 simulation results.
\end{frame}

\begin{frame}{Photon spectra on linear scale}
\begin{columns}[T,totalwidth=\textwidth]
\begin{column}{0.5\textwidth}
\begin{figure}
\centering
\includegraphics[width=\linewidth]{\WTaDir/photon_source_spectrum_linear.png}
\caption{Linear-scale photon spectrum for W-Ta.}
\end{figure}
\end{column}
\begin{column}{0.5\textwidth}
\begin{figure}
\centering
\includegraphics[width=\linewidth]{\UMoDir/photon_source_spectrum_linear.png}
\caption{Linear-scale photon spectrum for U-Mo.}
\end{figure}
\end{column}
\end{columns}
\tiny Source: own Geant4 simulation results.
\end{frame}

\begin{frame}{Neutron output on linear scale}
\begin{columns}[T,totalwidth=\textwidth]
\begin{column}{0.5\textwidth}
\begin{figure}
\centering
\includegraphics[width=\linewidth]{\WTaDir/neutron_source_spectrum_linear.png}
\caption{Linear-scale neutron spectrum for W-Ta.}
\end{figure}
\end{column}
\begin{column}{0.5\textwidth}
\begin{figure}
\centering
\includegraphics[width=\linewidth]{\UMoDir/neutron_source_spectrum_linear.png}
\caption{Linear-scale neutron spectrum for U-Mo.}
\end{figure}
\end{column}
\end{columns}
\tiny Source: own Geant4 simulation results.
\end{frame}

\begin{frame}{Neutron side-leakage pattern}
\begin{columns}[T,totalwidth=\textwidth]
\begin{column}{0.5\textwidth}
\begin{figure}
\centering
\includegraphics[width=\linewidth]{\WTaDir/h2_neutron_exit_side_surface.png}
\caption{Side-leakage map for W-Ta target.}
\end{figure}
\end{column}
\begin{column}{0.5\textwidth}
\begin{figure}
\centering
\includegraphics[width=\linewidth]{\UMoDir/h2_neutron_exit_side_surface.png}
\caption{Side-leakage map for U-Mo target.}
\end{figure}
\end{column}
\end{columns}
\tiny Source: own Geant4 simulation results.
\end{frame}

\begin{frame}{Hydrogen gas-production proxy by plate}
\begin{columns}[T,totalwidth=\textwidth]
\begin{column}{0.5\textwidth}
\begin{figure}
\centering
\includegraphics[width=\linewidth]{\WTaDir/h1_gas_h_plate.png}
\caption{Hydrogen-production indicator for W-Ta.}
\end{figure}
\end{column}
\begin{column}{0.5\textwidth}
\begin{figure}
\centering
\includegraphics[width=\linewidth]{\UMoDir/h1_gas_h_plate.png}
\caption{Hydrogen-production indicator for U-Mo.}
\end{figure}
\end{column}
\end{columns}
\tiny Source: own Geant4 simulation results.
\end{frame}

\begin{frame}{Helium gas-production proxy by plate}
\begin{columns}[T,totalwidth=\textwidth]
\begin{column}{0.5\textwidth}
\begin{figure}
\centering
\includegraphics[width=\linewidth]{\WTaDir/h1_gas_he_plate.png}
\caption{Helium-production indicator for W-Ta.}
\end{figure}
\end{column}
\begin{column}{0.5\textwidth}
\begin{figure}
\centering
\includegraphics[width=\linewidth]{\UMoDir/h1_gas_he_plate.png}
\caption{Helium-production indicator for U-Mo.}
\end{figure}
\end{column}
\end{columns}
\tiny Source: own Geant4 simulation results.
\end{frame}

\begin{frame}{Radiation-damage proxy by plate}
\begin{columns}[T,totalwidth=\textwidth]
\begin{column}{0.5\textwidth}
\begin{figure}
\centering
\includegraphics[width=\linewidth]{\WTaDir/h1_niel_plate.png}
\caption{NIEL-based damage indicator for W-Ta.}
\end{figure}
\end{column}
\begin{column}{0.5\textwidth}
\begin{figure}
\centering
\includegraphics[width=\linewidth]{\UMoDir/h1_niel_plate.png}
\caption{NIEL-based damage indicator for U-Mo.}
\end{figure}
\end{column}
\end{columns}
\tiny Source: own Geant4 simulation results.
\end{frame}

\begin{frame}{Neutron heatmap evolution (animation)}
\small
\begin{center}
\IfFileExists{\WTaDir/plate_neutron_heatmap_1.png}{
  \animategraphics[loop,autoplay,width=0.48\linewidth]{1}{\WTaDir/plate_neutron_heatmap_}{1}{7}
}{
  \fbox{\parbox[c][0.28\textheight][c]{0.46\linewidth}{\centering Placeholder: heatmap frame sequence (W-Ta)}}
}
\hfill
\IfFileExists{\UMoDir/plate_neutron_heatmap_1.png}{
  \animategraphics[loop,autoplay,width=0.48\linewidth]{1}{\UMoDir/plate_neutron_heatmap_}{1}{12}
}{
  \fbox{\parbox[c][0.28\textheight][c]{0.46\linewidth}{\centering Placeholder: heatmap frame sequence (U-Mo)}}
}
\end{center}
\vspace{-0.2em}
\begin{center}
\footnotesize Figure: temporal evolution of plate-level neutron field distribution.
\end{center}
\tiny Source: own Geant4 simulation results.
\end{frame}

\begin{frame}{Conclusion}
\small
\begin{itemize}
  \item The comparison framework provides physically consistent evidence for material selection.
  \item U-Mo demonstrates higher neutron productivity indicators in this simulation campaign.
  \item Final selection must balance neutron gain with thermal and radiation-damage constraints.
\end{itemize}
\end{frame}

\begin{frame}{References}
\tiny
\begin{itemize}
  \item Geant4 Collaboration, \textit{Geant4---a simulation toolkit}, NIM A 506 (2003) 250--303.
  \item ADS and NSC KIPT literature corpus summarized in project reference notes.
  \item Benchmark values and historical context from reviewed published sources.
  \item Numerical and visual results: own Geant4 simulation campaign.
\end{itemize}
\end{frame}

\end{document}
