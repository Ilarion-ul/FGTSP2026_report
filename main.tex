% !TEX program = xelatex
\documentclass[aspectratio=169,10pt]{beamer}

\usetheme{metropolis}
\usefonttheme{professionalfonts}

\usepackage{polyglossia}
\setdefaultlanguage{english}
\usepackage{fontspec}
\setmainfont{Arial}
\setsansfont{Arial}
\setmonofont{Courier New}

\usepackage{amsmath, amssymb, physics}
\usepackage{graphicx}
\usepackage{siunitx}
\sisetup{detect-all}

\title{Comparative Assessment of Spallation Target Materials}
\subtitle{Neutron Source NSC KIPT: A Geant4 Simulation Study}
\author{Ilarion Ulych}
\institute{Taras Shevchenko National University of Kyiv}
\date{\today}

\begin{document}

\begin{frame}[plain]
  \titlepage
\end{frame}

\begin{frame}{Talk roadmap (5 minutes)}
\small
\begin{enumerate}
  \item NSC KIPT project context and why this study matters.
  \item Physics principles: \(e^- \rightarrow \gamma \rightarrow (\gamma,n) \rightarrow n\).
  \item Geant4 model and engineering assumptions.
  \item Comparative results for W-Ta and U-Mo targets.
  \item Practical recommendation and next steps.
\end{enumerate}
\end{frame}

\begin{frame}{Project context and status}
\small
\begin{itemize}
  \item Goal: maximize neutron yield while controlling heat load and material damage in a compact electron-driven source.
  \item Legacy: NSC KIPT neutron source was designed as a multipurpose facility for research and applications.
  \item Current focus: simulation-based optimization under constrained operation and maintenance conditions.
  \item This presentation reports a \textbf{physics-consistent comparison} of candidate target materials.
\end{itemize}
\vspace{0.3em}
\centering
\includegraphics[width=0.60\linewidth]{images/Acselerator Blueprint.png}

\tiny Source: project blueprint image from repository.
\end{frame}

\begin{frame}{Physics principles behind neutron production}
\small
\begin{itemize}
  \item Primary process chain in the model:
  \[
    e^- \rightarrow \gamma \rightarrow (\gamma,n) \rightarrow n
  \]
  \item High-energy electrons generate bremsstrahlung photons in target plates.
  \item Photonuclear interactions produce neutrons with broad angular and energy distributions.
  \item Key trade-off: higher neutron yield vs. higher local energy deposition, NIEL proxy, and gas production (H/He).
\end{itemize}
\vspace{0.3em}
\tiny Model summary source: \texttt{Data/physics\_model\_principles.md}.
\end{frame}

\begin{frame}{Implemented Geant4 model (what is actually simulated)}
\small
\begin{itemize}
  \item Reference physics list: \texttt{QGSP\_BIC\_HPT}, with configurable production cut.
  \item Event-level primary generator includes energy spread, spatial profile, angular divergence, halo, and tilt defects.
  \item Scoring in target plates: \texttt{edep}, neutron exits, surface crossings, NIEL proxy, and H/He gas proxies.
  \item Normalization basis: all KPIs are first reported \textbf{per primary electron}.
\end{itemize}
\vspace{0.3em}
\tiny Source: \texttt{Data/physics\_model\_principles.md}.
\end{frame}

\begin{frame}{Target options and comparison KPIs}
\small
\begin{columns}[T,totalwidth=\textwidth]
\begin{column}{0.5\textwidth}
\textbf{Compared targets}
\begin{itemize}
  \item W-Ta
  \item U-Mo
\end{itemize}

\textbf{Main results requested}
\begin{itemize}
  \item \texttt{photons\_above5MeV\_per\_electron}
  \item \texttt{neutrons\_model\_exit\_per\_electron}
\end{itemize}
\end{column}
\begin{column}{0.5\textwidth}
\textbf{Important plots for interpretation}
\begin{itemize}
  \item neutron side-surface map
  \item plate neutron heatmap
  \item photon spectra (log and 4.5--30 MeV)
  \item neutron source spectrum (linear)
  \item NIEL and He-production by plate
\end{itemize}
\end{column}
\end{columns}
\vspace{0.3em}
\tiny Source: \texttt{Data/Important\_data}.
\end{frame}

\begin{frame}{Neutron transport visualization (W-Ta)}
\begin{columns}[T,totalwidth=\textwidth]
\begin{column}{0.5\textwidth}
\includegraphics[width=\linewidth]{Data/20260211_172835_W-Ta/h2_neutron_exit_side_surface.png}
\end{column}
\begin{column}{0.5\textwidth}
\includegraphics[width=\linewidth]{Data/20260211_172835_W-Ta/neutron_source_spectrum_linear.png}
\end{column}
\end{columns}
\vspace{0.2em}
\tiny Left: side-surface neutron exit map. Right: source neutron spectrum (linear scale).
\end{frame}

\begin{frame}{Photon field and plate-wise effects (W-Ta)}
\begin{columns}[T,totalwidth=\textwidth]
\begin{column}{0.5\textwidth}
\includegraphics[width=\linewidth]{Data/20260211_172835_W-Ta/photon_source_spectrum_4p5_30.png}
\end{column}
\begin{column}{0.5\textwidth}
\includegraphics[width=\linewidth]{Data/20260211_172835_W-Ta/h1_niel_plate.png}
\end{column}
\end{columns}
\vspace{0.2em}
\tiny Left: photon spectrum in 4.5--30 MeV range. Right: NIEL proxy per plate.
\end{frame}

\begin{frame}{U-Mo example: heatmap and gas production}
\begin{columns}[T,totalwidth=\textwidth]
\begin{column}{0.5\textwidth}
\includegraphics[width=\linewidth]{Data/20260212_072836_U-Mo/plate_neutron_heatmap_1.png}
\end{column}
\begin{column}{0.5\textwidth}
\includegraphics[width=\linewidth]{Data/20260212_072836_U-Mo/h1_gas_he_plate.png}
\end{column}
\end{columns}
\vspace{0.2em}
\tiny Left: neutron heatmap at plate level. Right: He-gas proxy by plate.
\end{frame}

\begin{frame}{Preliminary conclusion and recommendation}
\small
\begin{itemize}
  \item The comparison should be finalized by a single summary table with all KPIs normalized per primary electron.
  \item Material selection must balance: neutron productivity, thermal load distribution, and long-term radiation damage risk.
  \item Next immediate step: add numeric values from \texttt{particle\_yields\_per\_electron.json} for both targets directly to this deck.
\end{itemize}
\vspace{0.5em}
\textbf{Actionable TODO (next edit):}
\begin{itemize}
  \item insert final KPI table (W-Ta vs U-Mo),
  \item add uncertainty/comment on model limitations,
  \item freeze final references slide.
\end{itemize}
\end{frame}

\begin{frame}{References}
\tiny
\begin{itemize}
  \item Geant4 collaboration: \textit{Geant4---a simulation toolkit}, NIM A 506 (2003) 250--303.
  \item Geant4 physics-list documentation (QGSP\_BIC\_HPT).
  \item Internal project model notes: \texttt{Data/physics\_model\_principles.md}.
  \item Internal run outputs: \texttt{Data/20260211\_172835\_W-Ta/*}, \texttt{Data/20260212\_072836\_U-Mo/*}.
\end{itemize}
\end{frame}

\end{document}
